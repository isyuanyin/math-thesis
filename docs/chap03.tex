\chapter{相关工作}
\label{cha:sysu-thesis-latex-install-guide}

本章介绍与本文类似的相关工作,包括元学习的早期工作、针对小样本的一些元学习方式以及强化学习的各类元学习框架。


% 介绍早期的元学习的研究思想
\section{比较早期的工作}
学习如何去学习(learning to learn),有时候也称为元学习(meta learning)\cite{MetaLearningComputer2021} 在早期工作里已经有各式各样的讨论。例如改进遗传程序设计\cite{ohDiscoveringReinforcementLearning2020} \cite{schmidhuberEvolutionaryPrinciplesSelfreferential1987},学习神经网络更新规则 \cite{bengioLearningSynapticLearning1990},学习率适应 \cite{suttonAdaptingBiasGradient1992} 和领域不变知识的转移\cite{thrunLearningOneMore1994}等等。此类工作表明,不仅可以学习优化固定目标,而且可以改进在元级别进行优化的方式,从而解决人为设计学习规则不是完全适应问题本身的难题。


% 介绍普适性的元学习方法
\section{针对小样本的元学习}
对于小样本情形的机器学习,需要更高层次地设计,早期的一些工作从各个方面进行了研究。
Vinyals等人\cite{vinyalsMatchingNetworksOne2016b}(2016年)设计了可以学习一个网络结构的框架,有效地提升了图像分类任务的少样本学习的准确率。
Santoro 等人\cite{santoroMetaLearningMemoryAugmentedNeural2016}(2016年)设计了可以将过去的学习信息存储的增强记忆神经网络,该网络可以快速吸收新的数据,并利用此数据仅需几个样本就可以做出准确的预测。
Finn等人提出的 MAML(2017年)\cite{finnModelagnosticMetalearningFast2017} 和后来提出更通用的版本 \cite{finnMetaLearningUniversalityDeep2018}(2018年),通过先验的优化求解过程寻找更加适合求解优化问题的初始参数。
2017年 Duan 等人提出的 RL$^2$ 算法 \cite{duanRLFastReinforcement2016} 针对学习过程中实验次数过多问题,通过在代理的整个生命周期中展开LSTM,将学习本身描述为RL问题,从而利用过去的学习经验,减少在当前环境的训练次数。
所有这些方法都没有明确区分智能体和算法,所以,这些元学习算法也适用于单个智能体的程序结构。

% 单一环境任务下的元学习
\section{针对单一任务的元学习}
Xu等人\cite{xuMetagradientReinforcementLearning2018} (2018年)引入了元梯度RL方法;它使用反向传播遍历代理程序的更新,以计算相对于更新的元参数的梯度。
该方法已应用于元学习各种形式的算法组件,例如折扣因子,内在奖励,辅助任务,回报,辅助策略更新等\cite{ohDiscoveringReinforcementLearning2020}。然而,本文的工作与该方法的侧重点不一样,类似于LPG \cite{ohDiscoveringReinforcementLearning2020}:发现对更广泛的智能体和环境类别有效的通用算法,而不是适应特定的环境。

\section{元强化学习算法}
Houthooft 等人 在EPG \cite{houthooftEvolvedPolicyGradients2018} (2018年)使用进化策略来找到策略更新规则。
郑等人 \cite{zhengWhatCanLearned2020}(2018)提出将奖励函数本身作为强化学习算法的表示的元学习框架,表明勘探的一般知识可以通过奖励函数的形式获得。
但是,先前的工作最多只能将相同领域中的相似任务概括化。
最近,提出了MetaGenRL \cite{kirschImprovingGeneralizationMeta2019} 来元学习领域不变的策略更新规则,该规则能够从几个MuJoCo环境推广到其他MuJoCo环境。
然而很少有先前的工作试图发现完整的更新规则;取而代之的是,它们全都依赖于价值函数(可以说是RL的最基本构建块)进行引导。
DeepMind公司在2020年提出的LPG元学习自举机制\cite{ohDiscoveringReinforcementLearning2020},该论文尝试使用LSTM模型学习到一个强化学习算法,而不是使用手工的方式定义一个梯度策略更新规则。该方法在特定的环境下表现比经典的优势动作者-评论者(advantage actor-critick,A2C)算法 \cite{mnihAsynchronousMethodsDeep2016} 的表现要好。然而该算法过于激进,完全不考虑人类知识对算法的改进。本文类似于此方法,不同的是,本文的方法将融合A2C算法对智能体学习的指导作用。