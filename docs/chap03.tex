\chapter{算法}
\label{cha:usage-example}

\section{算法架构}

为了能让机器自动寻找到合适的强化学习算法,需要设计一个用参数表示的强化学习的策略更新算法。而为了使得用参数表示的强化学习算法能够具有更大的表示能力,本文设计一个能表示更丰富信息的向量,这个向量蕴含了智能体策略的信息,也能够在在设计的策略更新算法上更好地记忆在某类环境中更适合的学习方式。假设将新的策略更新算法参数化为 $\eta$。在这里,更新规则$\eta$并不直接更新策略$\pi$,而是直接更新预测向量$y_\theta$,然后通过一个简单解析器,从$y_\theta$解析出策略$\pi$来。

设环境的分布为 $p(\mathcal{E})$ 和智能体的初始参数为 $\theta_0$,则元学习的目标是:
\begin{equation}
    \eta^* = \arg \max \mathbb{E}_{\mathcal{E} \thicksim p(\mathcal{E})} \mathbb{E}_{\theta_0 \thicksim p(\theta_0)} [G]
    \label{eq:target}
\end{equation}
这里的 $G=\mathbb{E}_{\pi_N } [\sum_t^\infty \gamma^t r_t]$ 是智能体与环境交互的整个生命周期的期望回报。直观地,该目标旨在找到一个更新规则 $\eta$,以便在将其用于更新代理的参数直到其生命周期结束$ (\theta_0 \rightarrow \theta_1 \rightarrow \dots \rightarrow \theta_N)$ 时,该智能体会在给定环境中最大程度地提高预期收益。这个架构类似于LPG \cite{ohDiscoveringReinforcementLearning2020}。但不同的是,LPG事实上是利用策略梯度进行元学习更新参数 $\eta$,这样的元学习框架依赖于$\pi$的维度。本文将在元学习和智能体策略更新的时候都使用这样的方法。另一方面是,本文直接令预测的变量 $y$ 包含 $\pi$ 的信息,而不是让预测向量$y$和策略$\pi$作为同等的输入进行学习。总之,本文算法使用的是由元参数$\eta$参数化的更新规则,该更新规则要求智能体 $m$ 维分类预测向量 $y_\theta(s) \in \mathcal{R}^m$ ,然后在预测向量 $y$ 中解析出策略$\pi_y(a|s)$。事实上,元学习得到的更新规则由一个LSTM网络 \cite{sakLongShortTermMemory2014} 表示,记录如何从智能体与环境交互的轨迹中生成预测向量,然后再生成更新策略$\pi_{y}(a|s)$。这里的$y_\theta(s)$的更新过程和$\eta$的元学习过程如\autoref{fig:arch}所示,图中右边的不同深度的卡片代表着元学习过程$\eta$的迭代,而里面的$ (\theta_0 \rightarrow \theta_1 \rightarrow \dots \rightarrow \theta_N)$ 表示智能体策略更新过程中,智能体参数 $\theta$ 的迭代更新过程;左边的小图是预测向量$y$在LSTM模型中的更新过程。示意图中没有将算法中隐含的A2C算法部分展示出来,本文将在后面的算法描述中详细介绍。

更具体地,在每一个时间 $t$,输入是 $x_t = [r_t, d_t, \gamma, \pi(a_t | s_t), y_{\theta}(s_t), y_{\theta}(s_{t+1})]$。其中$r_t$是奖励;$d_t$ 二进制的数值,表示事件是否终止;而$\gamma$是折扣因子。实际学习过程中的将输入的$r_t$进行累积,计算出$G_t$,作为元学习和智能体策略更新的输入。与RL$^2$\cite{duanRLFastReinforcement2016}算法不同,本文算法不会将观测(状态)空间和动作空间作为输入,因此它不变。取而代之的是,它仅采用比所选动作 $\pi(a|s)$ 的概率含义更宽泛的预测向量$y$和回报值,这样可以通过挖掘智能体与环境交互过程所获得的奖励情况,找到一个可能更合适的策略更新规则。这种方式允许本文算法的架构可以应用于完全不同的环境,同时防止过度拟合。

\begin{figure}[h!]
    \centering
    \includegraphics[width=0.9\textwidth]{image/chap03/arch.pdf}
    \caption{算法架构示意图}
    \label{fig:arch}
\end{figure}



\section{智能体(策略)更新}
智能体的策略使用策略函数$\pi$进行表示,如上面的架构所述,$\pi$ 是由预测向量$y$来生成的,而$y$通过元参数$\eta$进行迭代求解出来。元学习和具体学习的过程都是固定住$\eta$,然后通过$\eta$决定的$\theta$的更新规则进行更新$\theta$,不同的是,元学习过程会在某个时刻更新$\eta$的值,在考虑智能体更新的算法中先忽略这一点。假设固定了策略更新算法 $\eta$ 后,通过梯度下降的优化方式更新智能体参数。这里使用的优化的损失函数如下:
\begin{equation}
    L(\theta) = \mathbb{E}_{\pi} \left[\sum_t -(G_t - v(s_t) ) \ln \pi_{y_t} (a_t|s_t) + \alpha ||\hat{y_{t+1}} - y_t+1||^2 \right]
    \label{eq:agent-update}
\end{equation}
其中,$\alpha$是比例参数,上面公式里面的符号省略了需要更新的参数 $\theta$,但其中的$y$是依赖于$\theta$,而策略$\pi$通过一个解析器间接依赖于$\theta$。在较高的层次上,$y$的更新过程指定应如何调整动作概率$\pi(s)$,并直接影响智能体的行为。智能体更新的过程如 \autoref{algo:policy-update}所示,里面的过程隐藏了解析器和解析器的更新方式。\autoref{eq:agent-update}的左边是策略梯度的损失函数,而右边是通过LSTM对预测向量的调整。从这个损失中可以看出,本文的方法如前面所言,是在传统的强化学习策略更新规则上,加上数据驱动方式自适应调整。而其中的自适应过程就是在下一个小节中所讲的元学习过程。$L$中右式的左边是$A2C$算法的公式,在实践中是用$n$步的收益和下一步的$v_{t+n}$估计计算估计的$G_t$。

对于策略解析器,可以使用复杂的多层感知机(MLP)模型或者线性模型与非线性模型的混合,这样可以使得解析器具有更加强大的表示能力。但是这样的方式可能会使得实践上模型的收敛变得困难。鉴于这篇文章更重要的目的是检验这个算法的可行性,这里只是单纯地使用了一个简单地线性模型,即策略$\pi$是预测向量$y$的一个线性组合,如\autoref{eq:pred2policy}所示,其中$\sigma$是softmax函数,是为了得到一个概率分布的向量。另外,简单的线性模型的解析器可以蕴含了策略梯度和优势动作者-评论者算法模型的表达能力。需要注意的是,本文是希望提出一个可改进的框架,一方面可以在这个框架的基础上进行调整;另一方面可以将框架里面的组件替换成更庞大且更容易训练的模型,这样就能够不断丰富该算法。线性模型只是时间和算力的一个折中的选择,而理论上,其它复杂的模型可能更加适合。
\begin{equation}
    \pi = \sigma (Wy)
    \label{eq:pred2policy}
\end{equation}

从$y$到$\pi$的策略解析器决定了预测向量与策略之间的关系,但是我们并不知道预测向量如何决定策略,也不知道预测向量如何替代策略进行迭代更新。所以需要调整解析器的表达方式,简单来说,即通过损失\autoref{eq:agent-update}的梯度下降调整解析器参数$W$的值。但是为了能达到收敛,$W$的更新不能太频繁,仅在设定$M$步后更新$W$。

\begin{algorithm}[h]
    \KwIn{$p(\mathcal{E})$:环境的分布, $p(\theta_0)$:初始化参数的分布}
    采样环境和智能体的参数 $\{ \mathcal{E} \sim p(\mathcal{E}), \ \theta \sim p(\theta_0)\}$
    \Repeat{$\theta$ 收敛} {
        智能体与环境交互,生成如序列 \ref{eq:mdp-sequence}的轨迹 \\
        计算$\theta$的梯度和$W$的累积梯度\\
        使用元参数 $\eta$ 和 \autoref{eq:agent-update} 更新 $\theta$的值 \\
        每$M$步,利用累积梯度更新$W$的值\\
        \If{生命周期结束} {
            重置环境 $\mathcal{E}$
        }
    }

    \caption{策略更新算法}
    \label{algo:policy-update}
\end{algorithm}



\section{元学习}

在进行元学习的训练时,将智能体放在不同类型的环境进行交互,然后更新元参数$\eta$,最后学习到一个策略更新规则。但是,这里不同类型的环境应该具有某些共性或者挑战,例如不能完全观测状态、长延迟收益和额外的干扰等,这也是自适应策略更新算法要弥补经典强化学习算法的关键地方。所以,环境分布 $p(\mathcal{E})$ 表示的是某一类具有相似挑战的环境。通过元学习得到的$\eta$应用于智能体与环境的交互中,就能将$\theta$更新到一个接近最优的取值。由于这里并没有将环境的状态作为输入,所以具有更好的泛化能力。

具体来说,通过将智能体与环境交互,然后不断将$\theta$更新到接近最优的$\theta_N$。为了将$\theta$引导到使得智能体交互过程回报$G$最大的值,这里使用如下 \autoref{eq:meta-gradient} 计算的梯度进行梯度上升更新$\eta$的值:
\begin{equation}
    \Delta \eta  \varpropto \mathbb{E}_{\mathcal{E}} \mathbb{E}_{\theta_0} \left[ \nabla \ln \pi_{N} (a|s) G \right]
    \label{eq:meta-gradient}
\end{equation}
其中,符号$y$隐含在$\pi$中,没有写出。这里是希望最后找到的$\eta$,它所代表的$\theta$更新方式能够使得智能体达到最优策略。其算法过程如\autoref{algo:meta-learning}所示。 在实践中,只要经过$K$步,就更新一次$\eta$。

\begin{algorithm}[h!]
    \KwIn{$p(\mathcal{E})$:环境的分布, $p(\theta_0)$:初始化参数的分布}
    
    初始化元参数 $\eta$
    采样批量的环境和智能体的参数 $\{ \mathcal{E} \sim p(\mathcal{E}), \ \theta \sim p(\theta_0)\}_i$

    \Repeat{$\eta$ 收敛} {
        \ForEach{$\mathcal{E}, \theta$}{
            使用参数 $\eta$ 更新 $\theta$ \\
            利用 \autoref{eq:meta-gradient} 计算 $\eta$ 的累积梯度 \\
            \If{生命周期结束} {
                更新$\eta$的值 \\
                重置环境 $\mathcal{E}$
            }
        }
    }
    \caption{元学习算法}
    \label{algo:meta-learning}
\end{algorithm}
