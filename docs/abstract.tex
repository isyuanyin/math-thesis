%%
% 摘要信息
% 本文档中前缀"c-"代表中文版字段, 前缀"e-"代表英文版字段
% 摘要内容应概括地反映出本论文的主要内容,主要说明本论文的研究目的、内容、方法、成果和结论。要突出本论文的创造性成果或新见解,不要与引言相 混淆。语言力求精练、准确,以 300—500 字为宜。
% 在摘要的下方另起一行,注明本文的关键词(3—5 个)。关键词是供检索用的主题词条,应采用能覆盖论文主要内容的通用技术词条(参照相应的技术术语 标准)。按词条的外延层次排列(外延大的排在前面)。摘要与关键词应在同一页。
% modifier: 黄俊杰(huangjj27, 349373001dc@gmail.com)
% update date: 2017-04-15
%%

\cabstract{
近些年来,强化学习借助深度学习技术在学习一些复杂的行为方面取得了很大的成功。经典的强化学习算法利用人工设计的策略更新方式,这些更新方法会涉及一些人为设计的概念,例如值函数、策略梯度等。这些概念和方法是在理想的强化学习假设提出来的,也就是马尔可夫决策过程。但是在实践中,这种假设并不完全符合实际问题;另一方面,在使用深度学习技术以及强化学习常用迭代方法,传统的方法并不一定能够取得理想中的效果。关于寻找更加适合实际问题的强化学习算法,也称为元强化学习,在很多前人的工作中考虑到。较近的一个工作是LPG,它尝试寻求一种与环境状态无关的算法;但是这种方法完全丢弃了人类的经验。本文希望设计一种寻求策略更新算法,在利用人类知识的同时,也使用数据驱动调整传统的方法,直到找到更适合问题的新的算法。换言之,本文设计了一种发现强化学习算法的机制。在实验中发现,这种方法具有一定的探索价值。
}
% 中文关键词(每个关键词之间用“,”分开,最后一个关键词不打标点符号。)
\ckeywords{强化学习,元学习,策略更新}

\eabstract{
% 英文摘要及关键词内容应与中文摘要及关键词内容相同。中英文摘要及其关键词各置一页内。
In recent years, reinforcement learning has achieved great success in learning some complex behaviors with the help of deep learning technology. Classical reinforcement learning algorithms use artificially designed strategy update methods. These update methods involve some artificially designed concepts, such as value functions, strategy gradients, and so on. These concepts and methods are proposed in the ideal reinforcement learning hypothesis, which is the Markov decision process. However, in practice, this assumption does not completely conform to the actual problem; on the other hand, in the use of deep learning techniques and common iterative methods of reinforcement learning, traditional methods may not be able to achieve the desired results. The search for reinforcement learning algorithms that are more suitable for practical problems, also known as meta-reinforcement learning, has been considered in the work of many predecessors. A recent work is LPG, which tries to find an algorithm that has nothing to do with the state of the environment; but this method completely discards human experience. This paper hopes to design an update algorithm for seeking reinforcement learning strategies. While using human knowledge, it also uses data-driven adjustment of traditional methods until a new algorithm that is more suitable for the problem is found. In other words, this paper designs esigns a mechanism for discovering reinforcement learning algorithms. In the experiment, it is found that this method has a certain exploratory value.
}
% 英文文关键词(每个关键词之间用,分开, 最后一个关键词不打标点符号。)
\ekeywords{Rienforcement Learning, Meta-Learning, Policy Updating}

