%%
% 背景知识
%%

\chapter{背景知识}
\label{cha:background-knowledge}

强化学习是为了处理现实中的问题而提出的概念。Sutton 在自己的著作 \cite{suttonReinforcementLearningIntroduction2018}中,将强化学习的理论框架抽象成马尔可夫决策过程(Markov Decision Program, MDP),这也是目前机器学习领域的共识 \cite{mnihHumanlevelControlDeep2015} \cite{mnihAsynchronousMethodsDeep2016} \cite{hesselRainbowCombiningImprovements2018}  ,本文也以此为基础介绍相关的概念和理论。这一章将从马尔可夫决策过程出发,解释强化学习的基本概念,以及介绍一些实践中遇到的问题及其经典的解决方法。

\section{基本概念}
\subsection{马尔可夫决策过程}

如第一章所言,我们假设智能体和环境的交互是离散的,如图 \ref{fig:agent-env-interaction} 所示, $t$ 的取值范围是 $\{0, 1, 2,3, \dots\}$。在每一个 $t$ 时刻依次发生如下事件:
\begin{itemize}
    \item 智能体观察处于状态 $S_t \in \mathcal{S}$ 的环境,得到观测 $O_t \in \mathcal{O}$,其中 $\mathcal{S}$ 是假设的状态空间,包含所有可能的状态;$\mathcal{O}$ 是假设的观测空间,包含所有可能的观测值。
    \item 智能体根据观测决定做出动作 $A_t \in \mathcal{A}$,其中 $\mathcal{A}$ 是动作空间。
    \item 在智能体做出动作后,环境会反馈一个奖励 $R_{t+1} \in \mathcal{R}$ 给智能体,其中$\mathcal{R}$ 是所有可能奖励的集合,一般假设 $\mathcal{R} \subset \mathbb{R}$。
\end{itemize}

为了叙述简单,假设智能体总是可以观测到环境的状态,即 $S_t = O_t$。智能体与环境交互的过程产生的序列如下:
\begin{equation}
    S_0, A_0, R_1, S_1, A_1, R_2, S_2, A_2, R_3,\dots
    \label{eq:mdp-sequence}
\end{equation}

在此基础上,进一步假设 奖励 $R_{t+1}$ 和下一个状态 $S_{t+1}$ 仅仅依赖当前的状态 $S_t$ 和动作 $A_t$,这称之为马尔可夫性质。用概率论的观点描述这种依赖就是:

\begin{equation}
    \mathbb{P}[S_t = s', R_t = r| S_{t-1}, A_{t-1} = a] = p(s', r | s, a)
    \label{eq:markov-condition-probability}
\end{equation}

恒满足等式 \ref{eq:markov-condition-probability} 的序列 \ref{eq:mdp-sequence} 就是马尔可夫决策过程的模型,表示 $p(s', r | s, a)$ 决定了所有时间上的状态转移和所得奖励的概率。在马尔可夫决策过程中,智能体选择一个策略(policy),在交互的每个时刻根据状态(观测)做出动作,用函数 $\pi : \mathcal{S} \times \mathcal{A} \rightarrow [0, 1]$ 表示策略,即选择某个动作的概率,
\begin{equation}
        \begin{split}
            \pi(a|s) = \mathbb{P} [A_t = a | S_t = s]
        \end{split}
\end{equation}

上述的模型可以导出一些更加直观的结论,
\begin{itemize}
    \item 状态转移概率:
    \begin{equation}
        p(s'|s, a) = \mathcal{P}[S_{t+1} = s' | S_{t} = s, A_{t} = a] = \sum_{r\in \mathcal{R}} p(s', r|s, a)
    \end{equation}
    \item “状态-动作”的期望奖励:
    \begin{equation}
        r(s, a) = \mathbb{E} [R_{t+1} | S_{t} = s, A_{t} = a] = \sum_{r\in \mathcal{R}} r \sum_{s' \in \mathcal{S}} p(s', r | s, a)
    \end{equation}
    \item “状态-动作-状态”的期望奖励:
    \begin{equation}
        r(s, a, s') = \mathbb{E} [R_{t+1} | S_{t} = s, A_{t-1} = a] = \sum_{r\in \mathcal{R}} r p(s', r | s, a)
    \end{equation}
\end{itemize}


\subsection{回报、折扣和值函数}
强化学习的目标是最大化累积的奖励,用回报(return)表示累积的奖励,则从 $t$ 时刻之后的回报 $G_t$ 定义为:
\begin{equation}
    G_t = \sum_{k \geq 1} R_{t+k} = R_{t+1} + R_{t+2} + R_{t+3} + \dots
\end{equation}

通过累积计算出来的回报往往是无穷大,通常会引入折扣(discount)$\gamma \in [0,1]$ 重新定义回报:
\begin{equation}
    G_t = \sum_{k \geq 1} \gamma^{k-1} R_{t+k} = R_{t+1} + \gamma R_{t+2} + \gamma^2 R_{t+3} + \dots
    \label{eq:return}
\end{equation}

对于给定的策略 $\pi$,可以计算状态 $S_t = s$ 的期望回报,记为
\begin{equation}
    v_\pi (s) = \mathbb{E}_\pi [G_t | S_t = s]
    \label{eq:state-value-function}
\end{equation}

当在状态 $s$ 采取一个确定的动作 $a$ 时,期望回报重新计算为如下,
\begin{equation}
    q_\pi(s, a) = \mathbb{E} [G_t | S_t = s, A_t = a] 
    \label{eq:action-value-function}
\end{equation}

上面两个函数分别称为状态价值函数和(状态-)动作价值函数。显然,两者的关系为 $ v_\pi(s) = \sum_{a} \pi(a|s) q_{\pi} (s, a)$。

\subsection{最优策略}
假设初始状态 $S_0 = s$ 的概率为 $p_0(s)$,则强化学习的求解目标自然是,
\begin{equation}
    \max_\pi \sum_s v_\pi(s) p_0(s)
    \label{eq:subject}
\end{equation}

而公式 \ref{eq:state-value-function} 的计算可以简化为各个状态 $v_\pi(s)$ 之间的计算关系,即
\begin{equation}
    v_\pi (s) = \sum_a \pi(a|s) \left[ \sum_{s', r} p(s', r | s, a)  [r + \gamma v_\pi(s')] \right]
    \label{eq:constraint}
\end{equation}

目标 \ref{eq:subject} 和 约束 \ref{eq:constraint} 与策略函数 $\pi(a|s)$ 自身的约束是一个线性规划问题。同样地,我们可以利用线性规划问题的求解方法求解该问题。不过假设 $\pi_*$ 是最优策略,而$v_*(s)$是该策略下的值函数,那么 $v_*(s)$ 应该满足

\section{实践中的强化学习}
\subsection{动态规划}
强化学习的动态规划算法主要在
% 在时间中全部

\subsection{蒙特卡洛方法}


\subsection{时序差分学习}

\subsection{逼近方法}
    % 如果 $Q$ 函数

\section{深度强化学习}
    % 近年来,深度学习随着现代计算机计算能力的提升而发挥出惊人的潜能。神经网络作为一个逼近器



