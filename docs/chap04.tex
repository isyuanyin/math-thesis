\chapter{算法}
\label{cha:usage-example}

\section{算法架构}
为了能让机器自动寻找到合适的强化学习算法,需要设计一个用参数表示的强化学习的算法,记为 $\eta$。设环境的分布为 $p(\mathcal{E})$ 和智能体的初始参数为 $\theta_0$,则元学习的目标是:

\begin{equation}
    \eta^* = \arg \max \mathbb{E}_{\mathcal{E} \thicksim p(\mathcal{E})} \mathbb{E}_{\theta_0 \thicksim p(\theta_0)} [G]
\end{equation}

这里的 $G=\mathbb{E}_{\pi_{\theta_N} } [\sum_t^\infty \gamma^t r_t]$ 是智能体与环境交互的整个生命周期的期望回报。这个架构类似于LPG \cite{ohDiscoveringReinforcementLearning2020}。但不同的是,LPG 利用策略梯度进行元学习更新参数 $\eta$,本文将结合策略梯度,将预测的变量 $y$ 包含 $\pi$ 的信息。

\section{智能体更新}



\section{元学习更新}
