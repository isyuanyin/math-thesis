\chapter{算法}
\label{cha:usage-example}

\section{算法架构}
为了能让机器自动寻找到合适的强化学习算法,需要设计一个用参数表示的强化学习的算法,记为 $\eta$。设环境的分布为 $p(\mathcal{E})$ 和智能体的初始参数为 $\theta_0$,则元学习的目标是:

\begin{equation}
    \eta^* = \arg \max \mathbb{E}_{\mathcal{E} \thicksim p(\mathcal{E})} \mathbb{E}_{\theta_0 \thicksim p(\theta_0)} [G]
\end{equation}

这里的 $G=\mathbb{E}_{\pi_{\theta_N} } [\sum_t^\infty \gamma^t r_t]$ 是智能体与环境交互的整个生命周期的期望回报。这个架构类似于LPG \cite{ohDiscoveringReinforcementLearning2020}。但不同的是,LPG 利用策略梯度进行元学习更新参数 $\eta$,本文将结合策略梯度,将预测的变量 $y$ 包含 $\pi$ 的信息。

\begin{figure}[h!] % image examples & compare
    \centering
    \includegraphics[width=0.4\textwidth]{image/chap04/simple_grid_world.pdf}
    \label{fig:compare1}
    \caption{复杂的两列对象的插入}
    \label{fig:complex}
\end{figure}

% \begin{figure}[h!] % image examples & compare
%     \begin{subfigure}{0.5\textwidth}
%         \includegraphics[width=0.5\textwidth]{image/chap04/simple_grid_world.pdf}
%         \label{fig:compare1}
%     \end{subfigure}
%     \begin{subfigure}{0.5\textwidth}
%         % \centering
%         \includegraphics[width=0.5\textwidth]{image/chap04/simple_grid_world.pdf}
%         \caption{右边的图像}
%         \label{fig:compare2}
%     \end{subfigure}
%     \caption{复杂的两列对象的插入}
%     \label{fig:complex}
% \end{figure}




\section{智能体更新}




\section{元学习更新}
